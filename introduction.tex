\section{Introduction}
Web page segmentation is the analysis process 
of dividing a web page into a coherent set of elements.
Examples of segments include sidebars,
headers, footers, to name a few. 
The basis of segmentation is that
the contents of a segment are perceived by the user
as perceptually similar. 
% Web page segmentation has been used 
% in a wide variety of testing and analysis problems, such as
% cross-browser testing~\cite{saar2016browserbite,huse2008using},
% mobile layout bugs testing and repair~\cite{mahajan2018automated,mahajan2018automated_intl},
% security testing~\cite{geng2015combating},
% and optimizing and directing crawlers~\cite{uzun2014effective,bharati2013higwget},
% to name a few.
% All these examples rely on page segmentation
% to perform their tasks.
% Segmentation provides a number of benefits in these
% scenarios, such as abstraction,
Segmentation provides a number of benefits,
such as page abstraction~\cite{uzun2014effective,bharati2013higwget}, 
localization of bugs and repairs~\cite{mahajan2018automated,mahajan2018automated_intl},
and page difference measurement~\cite{saar2016browserbite,huse2008using}.
% Accordingly,
% a reliable and accurate segmentation can have an impact on 
% many areas of software testing and analysis.

However, existing segmentation approaches 
have a number of drawbacks.
Document Object Model (DOM)-based techniques
are one way to perform segmentation
~\cite{rajkumar2012dynamic,vineel2009web,kang2010repetition}.
In this case, data is extracted from the DOM
and then various forms of analysis are performed to identify
patterns in the DOM.
While information gained from the DOM can be useful,
these approaches, however, have one key drawback.
The analysis performed is not necessarily related
to what the user is perceiving on screen,
and therefore the number of false positives
or false negatives can be high.
An alternative approach uses text-based information~\cite{kohlschutter2008densitometric, kolcz2007site}.
In this case, only textual nodes in the DOM are extracted
as a flat (i.e., non-tree) set of strings.
Various forms of analysis,
typically linguistic in nature,
are then applied to the textual data to identify
suitable segments.
While text and linguistic information is certainly
an aspect that the user can observe,
these approaches, by definition,
do not consider other important aspects of the page,
such as style, page layout and images.
Finally, another approach uses visual DOM properties 
to perform segmentation.
This is exemplified by the VIPS algorithm~\cite{cai2003vips},
a popular state-of-the-art segmentation technique~\cite{sleiman2013survey,campus2011web}.
Although VIPS stands for Vision-based Page Segmentation,
the technique only uses visual \emph{attributes} from the DOM 
(e.g., background color) in its analysis.
It does not perform a visual analysis of the page itself from a computer vision perspective,
such as analyzing the overall visual layout.
It also makes rigid assumptions about the design of a web page.
For instance, it assumes \code{<hr>} tags always behave as horizontal rules,
and therefore their approach segments the page when it sees that tag.
Such hard coded rules result in a fragile approach with reduced accuracy,
since developers often use tags in various non-standard ways
and combine them with various styling rules.
VIPS also requires a number of thresholds and parameters
that need to be provided by the user,
thereby increasing manual effort and reducing accuracy due to sub-optimal parameter tuning.

In this paper,
we propose a novel page segmentation approach, called \toolname, 
that combines DOM attributes and visual analysis to build features
and to provide a metric that guides clustering.
The segmentation process begins by an abstraction process that
filters and normalizes DOM nodes into abstract visual objects.
Subsequently, 
layout and formatting features are extracted from the objects. 
Finally, we build a visual adjacency neighborhood of the objects 
and use it to guide an unsupervised machine learning
clustering to construct the final segments.
Furthermore, \toolname is parameter-free,
requiring no thresholds for its operation and therefore
reduces the manual effort required and
makes the accuracy of the approach
independent of manual parameter tuning. 

We evaluate \toolname's segmentation effectiveness and efficiency
on 35 real-world web pages.
The evaluation compares \toolname with the 
state-of-the-art VIPS segmentation algorithm.
Overall, our approach is able to achieve 
an average of 156\% improvement in precision
and 249\% improvement
in F-measure, relative to the state-of-the-art.

This paper makes the following contributions:
\begin{itemize}
    \item A novel, parameter-free, segmentation technique that combines both the DOM and 
            visual analysis for building features and guiding an unsupervised clustering.
    \item An implementation of our approach,
          available in a tool called \toolname.
    \item A quantitative evaluation of \toolname in terms of
          segmentation effectiveness and efficiency.
\end{itemize}




